\documentclass[a4paper, 10pt]{extarticle}
%\documentclass[10pt]{}
\usepackage[top=0.2in, bottom=0.3in, left=0.3in, right=0.3in]{geometry}
\usepackage{multirow}
\usepackage{hyperref}
\usepackage[singlespacing]{setspace}

\hypersetup{ colorlinks, citecolor=green, filecolor=black, linkcolor=blue, urlcolor=blue }
\author{Pradeep Kashyap Ramaswamy}
\date{December 2016}
%\raggedbottom
%\flushbottom
\interlinepenalty=100000
\begin{document}


{\begin{center}
%Keeping the header in font size 10 which is normalsize.
\begin{tabular}{l c r}
    % \hline
    pradeep.kashyap@wisc.edu  & \multirow{3}{*}{\LARGE{\textbf{ Pradeep Kashyap Ramaswamy}}} & \ \ \ \ \ \ +1 (608) 504 1096 \\
     2110 University Ave, Apt 409 & & \href{https://www.linkedin.com/in/pradeep-kashyap-ramaswamy}{LinkedIn} \\
     Madison, WI -- 53726 & & \href{https://github.com/pradeep0605}{GitHub}\\
     %\hline
\end{tabular}
\end{center}}


\begin{flushleft}
{\LARGE{\textbf{E}}\Large{\textbf{DUCATION}}}
\end{flushleft}
\vspace{-0.1cm}\hrule \vspace{.1cm}

% Keeping the entire document in large font.
\begin{large}

%Master's degree.
{\begin{tabular}{ p{3cm} p{10cm} p{4.4cm} }
    %\hline
    \textbf{Madison, WI} & \vspace{-.6cm} \begin{center}\textbf{{University of Wisconsin---Madison}} \end{center} \vspace{-0.8cm} &
    \vspace{-0.6cm}
    \begin{flushright}
    \textbf{Sept 2016 - Dec 2017} 
    \end{flushright}
    \vspace{-0.8cm}\\
        \hline
\end{tabular}}


\begin{itemize}
    \setlength\itemsep{-0.05 cm}
    \item {\textbf{Master of Science} in \textit{Computer Science}, GPA: \textbf{4.0 / 4.0}}
    %\vspace{-.2cm}
    \item{\textbf{Courses}: OS, Topics in DBMS,  Advanced--OS, Algorithms, Data Science.}
\end{itemize}

%Bachelor's degree.
\hrule
\vspace{.1cm}
{\begin{tabular}{ p{3cm} p{10cm} p{4.4cm} }
    %\hline
    \textbf{Mysore, India} & \vspace{-.6cm} \begin{center}\textbf{{Sri Jaychamarajendra College of Engineering}} \end{center} \vspace{-0.8cm} &
    \vspace{-0.6cm}
    \begin{flushright}
    \textbf{Aug 2010 - May 2013} 
    \end{flushright}
    \vspace{-0.7cm} \\
    \hline
\end{tabular}}
%\vspace{-0.2cm}
\begin{itemize}
    \setlength\itemsep{-0.05 cm}
    \item {\textbf{Bachelor of Science} in \textit{Computer Science}, GPA: \textbf{9.49 / 10}}
    %\vspace{-.2cm}
    \item{\textbf{Courses}: Data Structures, Digital Design,  Computer Organization \& Architecture, OOP, Software Engineering, Algorithms, Graph theory, Compilers, Finite Automata, OS, Networks-- I \& II, Graphics.}
\end{itemize}
\vspace{.2cm}
%=================End of Education section ================
{\LARGE{\textbf{E}}\Large{\textbf{MPLOYMENT}}}
\vspace{.1cm} \hrule \vspace{.1cm}
{\begin{tabular}{ p{6cm} p{5cm} p{6.5cm} }
    %\hline
    \textbf{R\&D Engineer Software--I} & \vspace{-.6cm} \begin{center}\textbf{{Broadcom}} \end{center} \vspace{-0.8cm} & 
    \vspace{-0.7cm} 
    \begin{flushright} \textbf{July 2013 - Aug 2016} \end{flushright} 
    \vspace{-0.8cm}\\
    \hline
\end{tabular}}
\vspace{0.3cm} \\
Worked as Firmware Engineer on BRCM WiFi chipsets and as Linux driver developer on MIPI-DSI compliant display unit. My responsibilities included feature development, bug fixing and productization  of chipsets.
%\vspace{-0.1cm}
\begin{itemize}
    \setlength\itemsep{-0.05 cm}
    \item{Developed new features and fixed bugs in display driver for productization of  mobile devices that went into market with Samsung and HTC.}
    
    \item{Implemented and productized VSDB algorithms on 4358 and 4359 WiFi chipsets. Impact: VSDB--Virtual Simultaneous Dual Band--enabled multiple wireless network host interfaces on a single hardware.}
    
    \item{Bring up of VSDB in 43201 low-power chip, from ground up. Ownership of MSCH (Multi-Channel Scheduler) algorithms in 43012. MSCH--a central channel scheduler--bug fixes had great impact on other MSCH related teams and new chipsets. 43012 went into market as Samsung Gear 3's WiFi unit.}
   % \item{Implemented VSDB Soft-AP+STA, VSDB-RSDB (Real SDB) Mode-Switch and ASDB (Adaptive SDB) algorithms from ground up. Impact: Seamless switch between VSDB and RSDB. Product went into market with Samsung Note 5.}
    %\vspace{-.2cm}
    \item{Implemented Auto-SHM (Shared Mem) feature by converting configuration values in macros to structure members to avoid ROM abandons. Impact: Saved wastage of RAM due to code abandons in ROM. This resulted in lower RAM requirement in future chips and hence, it lowered cost per chip to the company.}
    %\vspace{-0.2cm}
\end{itemize}

\vspace{.1cm}
%=================End of Employment section ================
\begin{flushleft}
{\LARGE{\textbf{P}}\Large{\textbf{ROJECTS}}}
\end{flushleft}
\vspace{-0.1cm}\hrule %\vspace{.1cm}

\begin{itemize}
    \setlength\itemsep{-0.02 cm}
    \item{\textbf{Spark vs. Heron: Streaming Benchmark}|\href{https://github.com/rdamkondwar/Streaming-Benchmark}{GitHub}|10/2016 to 12/2016\textbf{{\Large{:}}} A benchmarking suite to compare the performance of  stream processing systems. Project concluded in results demonstrating Heron performing better than Spark. \textbf{Java}. Under the guidance of Prof. \href{http://pages.cs.wisc.edu/~jignesh/}{Jignesh M. Patel.}}
    
    %\vspace{-.1cm}
    
    \item{\textbf{Operating System Projects} in XV6 kernel|\href{https://github.com/pradeep0605/CS537-IntroToOS}{GitHub}|09/2016 to 12/2016\textbf{{\Large{:}}} \href{https://github.com/pradeep0605/CS537-IntroToOS/tree/master/Project2b}{1)} Multi-level Feedback Queue Scheduler. \href{https://github.com/pradeep0605/CS537-IntroToOS-Project2a}{2)} Bash-like Shell implementation \href{https://github.com/pradeep0605/CS537-IntroToOS-Project3b}{3)} Shared Pages support in kernel. \href{https://github.com/pradeep0605/CS537-IntroToOS/tree/master/Project4b}{4)} Kernel thread support \href{https://github.com/pradeep0605/CS537-IntroToOS/tree/master/Project5a}{5)} File System checker \href{https://github.com/pradeep0605/CS537-IntroToOS/tree/master/Project5b}{6)} Checksum protected file support. \textbf{C}.}
    %\vspace{-.1cm}
    
    \item{\textbf{Sudoku solver}|\href{https://github.com/pradeep0605/Sudoku_Solver}{GitHub}|04/2013\textbf{{\Large{:}}} Set-Operations ({$\cup$}, {$\cap$}, {$\setminus$})  based approach to solve Sudoku puzzles. With $n$ empty cells and $d X d$ grid (9x9 generally), recursive backtracking algorithm takes  $O(d^n)$ time. My algorithm took $((n^2 + n) / 2) * 6d$ \ $\Longrightarrow$ \ $O(n^2 *d)$ time. \textbf{C++, Java}.}  
    %\vspace{-.1cm}
    
    \item{\textbf{Basic SIC Assembler and Simulator}|\href{https://github.com/pradeep0605/SIC_assembler_and_simulator}{GitHub}|10/2011 to 11/2011\textbf{{\Large{:}}} This project provides basic environment to run the hypothetical SIC (Simplified Instructional Computer) programs. \textbf{C++}.} 
    %\vspace{-.1cm}
    
     \item{\textbf{Four Phased Image Compression}|\href{https://github.com/pradeep0605/FourPhasedImageCompressor}{GitHub}|06/2011 to 08/2011\textbf{{\Large{:}}} An Image (24-bit depth, RGB Image) compression tool which uses four pipelined stages from Compression: Aggregation, Bit Truncation, RLE and Huffman Coding. Compression of 84.37 \% upto 99\% was achieved. \textbf{C++}.} 
    %\vspace{-.1cm}
    
    \item{\textbf{Remote System Tracker and Controller}|\href{https://github.com/pradeep0605/RemoteSystemTrackerAndController}{GitHub}|11/2009 to 02/2010\textbf{{\Large{:}}} A monitoring application which tracks the activities and controls computers in LAN. Features include screen capture, file transfer and remote login. \textbf{Java}.} 
    %\vspace{-.1cm}
    
\end{itemize}
\vspace{.1cm}
%=================End of Project section ================
\vspace{-0.2cm}
\begin{flushleft}
{\LARGE{\textbf{S}}\Large{\textbf{KILLS}}}
\end{flushleft}

\vspace{-.1cm} \hrule \vspace{-.1cm}
%On a scale of 0 to 10, \textbf{\underline{B}}eginner:1-3, Prior Experience:4-5, Proficient:6-7, Expert:8-10

\begin{itemize}
    \setlength\itemsep{-0.03 cm}
    \item{\textbf{Languages}: C, C++, Java, Python.}
    
    \item{\textbf{Technology and Software}: GDB, MySQL, HTML \& CSS, SVN/Git, Wireshark, gnuplot, dotty, ctags, cscope, Shell Scripting, Heron, Java Servlets, \LaTeX{}.}
    
    \item{\textbf{Operating Systems}: Unix variants, Windows.}
    %\item{\textbf{Expert}: C.}
    %\item{\textbf{Proficient}: Java, Algorithms, Data Structures, , Linux, Windows.}
    %\item{\textbf{Prior Experience}: C++, Python, VB, GDB, MySQL, HTML \& CSS, SVN/Git, Wireshark, gnuplot, dotty, ctags, cscope, Java Servlets, \LaTeX.}
    
\end{itemize} 

\vspace{0.1cm}
\hrule \vspace{-.2cm}
\begin{center}
\tiny{| End of Resume |}
\end{center}

%%====================== END OF RESUME ==========================

\newpage

% ==================Additional Information Page =================
\begin{center}
 \large{\textit{This page contains additional information that is not  essential part of the resume.}}
\end{center}
\vspace{.2cm}
\begin{flushleft}
{\LARGE{\textbf{A}}\Large{\textbf{WARDS}} \large{AND} \LARGE{\textbf{A}}\Large{\textbf{CHIEVEMENTS}}}
\end{flushleft}
\vspace{-.2cm} \hrule \vspace{-.1cm}
\begin{itemize}
    \setlength\itemsep{-0.05 cm}
    \item{Secured $10^{th}$ rank in Bachelor's CET (Common Entrance Test) among 50,000 students (top 0.02 \%).}

    
    \item{Awarded \textbf{twice} with \textit{``Award of Recognition''} in Broadcom for the contributions towards VSDB \& RSDB (Real SDB) for 4358 \& 4359 chipsets and, Ownership of ASDB (Adaptive SDB) for 4359 chipset.}

    
    \item{Awarded with ``FedEx International Scholar of the Year 2016'' Scholarship, Mumbai. \href{https://www.linkedin.com/pulse/bangalore-student-wins-fedex-scholarship-edu-beanz}{News link}.}

    
    \item{Secured \textbf{First} Prize in the following competitions.
        \vspace{-0.2cm}
        \begin{itemize}
            %\setlength\itemsep{-0.0005 cm}
            \item{``Puzzle the Unpuzzled'' at IISc by Department of CSA, Bangalore, Mar 2012.}
           
            \item{``Top Coders'' at AIT, Coimbatore, by Department of CS, Sept 2012.}
            
            \item{C--coding conducted during Cyberia'12 by IEEE SJCE.}
            
            \item{Night out C-coding contest conducted during Cyberia'12 by IEEE SJCE.}
                        
            \item{C--coding conducted during FOSSCamp'11 by GNU/Linux Campus Club (LCC) SJCE.}
            
            \item{GLDB (GNU Linux Debugging) conducted during FOSSCamp`12 by LCC SJCE.}
            
            \item{Gaming (Open Arena) conducted during Even semester event 2013 by LCC SJCE.}
        \end{itemize}
    }
       
    \item{Secured other prizes in the following competitions.
        \vspace{-0.2cm}
        \begin{itemize}
            \setlength\itemsep{0.05cm}
            \item{\textbf{Second} Place in X-Files Paper presentation contest conducted during Technologix'12 by CSI SJCE.}
            
            \item{\textbf{Finalist} (top 7 teams out of 40) in Dennis Ritchie 'C'--coding competition at IISc by Department of Computer Science and Automation, 2012.}
        \end{itemize}
    }    
\end{itemize}
\vspace{.2cm}

\begin{flushleft}
{\LARGE{\textbf{M}}\Large{\textbf{INI}}\large{--}\LARGE{\textbf{P}}\Large{\textbf{ROJECTS}}}
\end{flushleft}
\vspace{-.2cm} \hrule \vspace{-.1cm}

\begin{itemize}
     \setlength\itemsep{-0.01 cm}
    \item{\textbf{8-bit and 16-bit Huffman Compressor}|\href{https://github.com/pradeep0605/HuffmanCompressor}{GitHub}|05/2011 to 06/2011\textbf{{\Large{:}}} A Generic file compression tool based on Huffman Coding which uses 8-bits and 16-bits as sampling lengths for symbols. \textbf{C++}.} 
    
    
    \item{\textbf{Online Coding Competition}|\href{https://github.com/pradeep0605/LCC-SJCE_OnlineCodingWebInterface}{GitHub}|03/2013\textbf{{\Large{:}}} An web interface created for conducting C--coding competitions as part of LCC--SJCE. \textbf{Java Servlets}.} 
    
    \item{\textbf{Self-Designed Generic Classes}|\href{https://github.com/pradeep0605/GenericClasses}{GitHub}|2011 to 2012\textbf{{\Large{:}}} This project contains two of the generic classes I have designed: Threads and String. Both of these classes can be used similar to Java Threads (Extend and override run method) and Java Strings (with concatenation, and other operator overloading) respectively. \textbf{C++}.} 
    
    \item{\textbf{Round-Robin Schedule Simulator}|\href{https://github.com/pradeep0605/RoundRobinScheduleSimulator}{GitHub}|12/2011\textbf{{\Large{:}}} A tools which simulates the schedules of a Round Robin process scheduler. \textbf{C++}.} 
\end{itemize}
\vspace{.2cm}

\begin{flushleft}
{\LARGE{\textbf{V}}\Large{\textbf{OLUNTEERING}} \large{AND} \LARGE{\textbf{L}}\Large{\textbf{EADERSHIP}}}
\end{flushleft}
\vspace{-.2cm} \hrule \vspace{-.1cm}

\begin{itemize}
    \setlength\itemsep{0.05cm}
    \item{A technical volunteer for LCC (Linux Campus Club)| A student body for promoting the use of open-source tools and technologies}
   
    \item{Conducted technical sessions to promote use of FOSS to students from other departments }
    
    \item{Have taught Computer Science courses like Digital Design, C Programming and Data Structures to Diploma (An associate degree) students.}
    
    \item{Conducted three coding competitions as part of LCC's FOSS camp and FOSS bytes events.}
    
    \item{Donated 20,000 INR every year to ``Hoysala Karnataka Sangha'' to aid under privileged children for their education.}
    
\end{itemize}

\end{large}
\end{document}
